\chapter{Solar System Radar with the GBT}\label{chap:radar}

\vspace{-1cm}

\section{Introduction}
The \gls{GBT} participates in radar observations of near-Earth asteroids and
comets, as well as Lunar and planetary mapping and rotation studies.  These
are done in collaboration with the Arecibo Observatory which can transmit at
2380~MHz (\gls{Sband}) or 430~MHz (\gls{Pband}), and with JPL/Goldstone at
8560~MHz (\gls{Xband}) or 7190~MHz (\gls{Cband}).

Anyone wishing to do radar studies should collaborate with Scientists at either
Arecibo or NASA/JPL to plan the experiment.  Proposals should be submitted to
NRAO-Green Bank as well as to Arecibo or JPL.  Opportunities for radar
observations can arise on short notice, in which case proposers can make use of
\dq{DTT} (director's discretionary time) proposals if the normal proposal
process is not timely enough.  Use the \gls{NRAO} proposal tool to submit all
proposals, and indicate the proposal is for \dq{DTT}; these proposals will be
reviewed immediately without waiting for the usual review process. 

\vspace{-0.5cm}
\section{Data Acquisition Backends}

\begin{table}[!b]
\begin{center}
\caption[Radar data acquisition backends]
{Radar data acquisition backends.
\label{table:radarbackends}}
\begin{tabular}{ccc}
\toprule
Backend & Sample rates & Bandwidths \\
\midrule
PFS & 5,10 MHz & 0.5--10.7 MHz \\
JPL & 6.25--160 MHz & 1.3--60 MHz \\
VEGAS & undefined as yet  & up to 1 Ghz \\
\bottomrule
\end{tabular}
\end{center}
\end{table}

There are two data acquision backends:
\vspace{-0.25cm}
\begin{enumerate}[label=\bfseries{\arabic*.},leftmargin=*,
labelindent=\parindent, itemsep=0pt]
\item The \gls{PFS} does either 2 or 4-bit sampling. Installed in 2001. Consult
Jean-Luc Margot or Bruce Campbell if you want to use it.
\item JPL system, installed in 2014, does 8-bit sampling.
\end{enumerate}
There is a possibility of a third option, using the \gls{VEGAS} spectrometer.
This is being developed and may be available at a later time. At present, the
best choice is the JPL system which can be configured flexibly under computer
control for a wide choice of bandwidths and sampling rates. The old \gls{PFS}
is configured by hand by changing cables and filters to its inputs. Sample
rates and bandwidths are listed in table~\ref{table:radarbackends} and
descriptions on the usage of both backends can be found in the following wiki pages:

\begin{description}[itemsep=0pt]
\item[PFS system:] \htmladdnormallink
{https://safe.nrao.edu/wiki/bin/view/GB/Observing/RadarObserverAdvice}
{https://safe.nrao.edu/wiki/bin/view/GB/Observing/RadarObserverAdvice}
\item[JPL system:] \htmladdnormallink
{https://safe.nrao.edu/wiki/bin/view/GB/Observing/JPLRadar}
{https://safe.nrao.edu/wiki/bin/view/GB/Observing/JPLRadar}
\end{description}


\newpage


\section{GBT Scheduling Blocks}
The two backends, \gls{PFS} and JPL, are connected in parallel, so the
following configuration works for either one.  It should be noted that
the data recording is not controlled through the \gls{GBT} user interface
(\gls{Astrid} ).  The \gls{SB} tracks the object, but the user has
to run the data acquisition process independently.

\noindent Here is an example script for 8560 MHz observations.

\lstinputlisting[language=PythonAstrid,
backgroundcolor=\color{sbBackground},
caption={[Example SB for radar observations]
Example SB for radar observations.},
label={lst:radar}]
{radar_sb.py}

The ephemeris file referred to in the
{\bfseries{\textcolor{pythonKeywords}{Catalog}}()}
command, above, gives the coordinates for the object, as described in the
next section. The object name, in this case \dq{1999JV6} is defined in the
file referred to in the {\bfseries{\textcolor{pythonKeywords}{Catalog}}}
command.  One may omit the {\bfseries{\textcolor{pythonKeywords}{Focus}}}
part of the {\bfseries{\textcolor{pythonKeywords}{AutoPeakFocus}}} command
when observing at the lower frequencies, i.e, 2380~MHz or 430~MHz.

Refer to Chapter~\ref{chap:scripts} for more information on \gls{GBT}
configurations and \glspl{SB}.

\newpage

\section{Tracking moving objects}
Here is an example of an ephemeris file for an asteroid. Refer to
\S~\ref{sec:ephemeris} for description of the \dq{Ephemeris format}.

\lstinputlisting[language=PythonAstrid,
backgroundcolor=\color{catalogBackground},
caption={[Example ephemeris file for an asteroid]
Example ephemeris file for an asteroid.},
label={lst:asteroidephem}]
{asteroidephemexample.astrid}

\noindent Consult \S~\ref{sec:comets} for a description of obtaining
ephemeris data from the NASA/JPL \dq{Horizons} website and converting
it for use with \gls{Astrid}.  Here is a brief description of the process:

\begin{itemize}[itemsep=1pt]
\item Access the JPL Horizons web interface: \htmladdnormallink
{http://ssd.jpl.nasa.gov/horizons.cgi}
{http://ssd.jpl.nasa.gov/horizons.cgi}
\item Set up Horizons web-interface as follows:
\begin{description}[itemsep=1pt]
\item[ephemeris type:] OBSERVER
\item[target body:] [select the object]  
\item[Observer Location:] Green Bank (GBT) [select from list of observatories]
\item[Time Span:] [put in desired values]
\item[Table Settings:] QUANTITIES=1,3,20\newline
(1) Astrometric RA\&Dec, (3)rates in RA\&Dec, and (20) Range and range rate
\item[Display/Output:] plain text
\end{description}
\item Click \dq{Generate Ephemeris}
\item Use the web browser file menu to save the output file as
(for example) {\tt cometfilename.txt}
\item Run the program {\tt jpl2astrid cometfilename.txt}\\
A new file with an {\tt.astrid} extension will be created. An example
of such a file is shown in Script~\ref{lst:asteroidephem}.
\end{itemize}

The resulting {\tt.astrid} file is used as an argument to the \gls{Astrid}
{\bfseries{\textcolor{pythonKeywords}{Catalog}}()} command.

\newpage

\noindent If you wish to track the velocity, use:
\begin{itemize}
 \item {\tt jpl2astrid cometfilename.txt vel}
 \end{itemize}
This will put the velocity in the {\tt.astrid} file.  This option is usually
not necessary because the relative velocity of the object is compensated by
the transmitter, i.e., the transmitted frequency at Arecibo or Goldstone is
programmed to result in a constant frequency received at Green Bank.

\noindent {\bf Note:} the coordinate rates, columns 5 and 6 in the above
example, as given by the Horizons listing, are:
\begin{itemize}
\item $dRA*cosD$
\item $d(DEC)/dt$
\end{itemize}
In converting to the {\tt.astrid} result, {\tt jpl2astrid} divides the RA rate by
cosine(Declination) so that it is the rate in the RA, rather than in
$RA*cos(Dec)$.  The units in both coordinates are arcseconds per hour.

The {\tt jpl2astrid} program often does not fill in the object's name correctly.
One should edit the NAME in the {\tt .astrid} file to be something meaningful,
and one should make sure the object name in the \gls{SB} matches that in
the ephemeris table.

