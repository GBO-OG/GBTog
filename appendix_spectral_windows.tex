\chapter{Introduction to Spectral Windows}\label{appendix:spectralwindows}

Several simultaneous frequency bands may be specified with a list of rest
frequencies and offsets (keywords {\bf restreq}, {\bf deltafreq}, see
\S~\ref{sec:keywords}). If using a backend other than \gls{VEGAS}, the
{\bf nwin} (number of spectral windows) keyword will also need to be specified.
Each spectral window includes both polarizations. i.e., if you specify one window,
you get two \glspl{IFsys} routed to the back end device, one for each
polarization; if you specify two windows, you get 4 glspl{IF}, and so forth.

The configuration software tries to put the midpoint of the total frequency range 
spanned by all windows at the center of the nominal IF1 band so as to use the 
narrowest \gls{IF} bandpass filters that will pass the desired range of frequencies.
In some uncommon cases this is not possible, so the \gls{IF} bandwidth must be
increased to pass the desired range of frequencies. For prime focus receivers, the
total \gls{IF} bandwidth is 240 MHz; for the Gregorian receivers, up to 4 GHz is
possible, depending on the receiver.

The user specifies the rest frequencies ({\bf restfreq} keyword) and may also
specify a range of radial velocities ({\bf vlow} and {\bf vhigh} keywords,
see Appendix~\ref{appendix:vlowhigh}). The various \gls{IF} filters are set to 
include the required range of frequencies in the local frame required by the 
radial velocity range. The configuration software predicts the local frequency for 
each spectral window based on the rest frequencies and the radial velocity. 
During observing the tracking \gls{LO} will correctly track the doppler tracking
frequecy set by the {\bf dopplertrackfreq} keyword.  If {\bf dopplertrackfreq} is
not provided, the default value will be the first spectral window specified by the
{\bf restfreq} keyword will be used (if not using the advanced restfreq syntax).
Because there is only one tracking \gls{LO}, the other spectral windows are set
up with frequency offsets in the local frame with respect to the doppler tracking
frequency. When observing at a variety of high velocities, one should run a
configuration for each change of velocity (i.e., do not rely on just changing the
velocity in the \gls{LOone} manager), and one should set {\bf vlow}={\bf vhigh}.  

Note that the {\bf deltafreq} keyword gives frequency offsets that are applied in 
the local (or topocentric) frame i.e., it is applied as an offset in the \gls{IFsys}.
For example, if $V_{frame}$ is velocity of the reference frame, $V$ is source velocity
in that frame, $\nu_{rest}$ is the rest frequency of the line and we use the Radio
definition of velocity then the topocentric frequency will be

\begin{equation}
\nu_{topo} = \nu_{rest} \left( 1 - {\left(V+V_{frame}\right) \over c} \right) 
+ deltafreq
\end{equation}

Finally note that the expert user may specify any of the \gls{IFsys} conversion 
frequencies and total \gls{IFsys} bandwidth, overriding the calculations done by the 
configuration software ({\bf ifbw}, {\bf if0freq}, {\bf lo1bfreq}, {\bf lo2freq}, and
{\bf if3freq} keywords). This option may be needed in some peculiar cases. Of course
one needs a good knowledge of the \gls{IFsys} to make use of this option. 

\section{Array Receiver Spectral Windows}

Array Receivers can be configured with a variety of spectral windows. The
{\tt configtool}, part of \gls{Astrid}, sets up these spectral windows, and a new
syntax was required to specify more complex configurations.  Each feed has the
potential to be tuned to a different rest frequency. For the \gls{KFPA} receiver,
a special \dq{all} beam mode is defined which uses all 7 beams, plus one beam tuned
to a second, different spectral window. This stretches the syntax of the
{\tt configtool} {\bf restfreq} and {\bf deltafreq} keywords. In order to support
these modes within the {\tt configtool}, expanded values and intepretations of
{\bf nwin}, {\bf deltafreq} and {\bf restfreq} were implemented.

The syntax uses a python dictionary for the {\bf restfreq} and {\bf deltafreq}
keyword values for \gls{KFPA} configurations. The {\bf restfreq} dictionary maps
beams and frequencies of the spectral windows. The delta frequency is a map of
{\bf deltafreq} to {\bf restfreq}. The list of values syntax continues to be
supported for simpler modes. When the dictionary is used to specify the rest
frequencies, this dictionary must contain a key named \sq{DopplerTrackFreq}. 
The value assigned to this key is the rest frequency that will be used by
the \gls{LO} as the Doppler tracking frequency.

\noindent The following examples show of how to specify {\tt configtool}
frequency settings:
\begin{itemize}[leftmargin=*]
\item {\bf Example 1}: Requests that beams 1,2,3 and 4 have a rest frequency of 24000~MHz,
that beams 5,6,7 have a rest frequency of 23400~MHz and the 2nd beam 1 \gls{IF} band has
a rest frequency of 25000~MHz. There are no delta frequencies used in this observation. 
For non zero delta frequencies, the {\bf deltafreq} values should be specified in the
same manor as the {\bf restfreq}.

\lstinputlisting[language=PythonAstrid,captionpos=b,frame=single,framerule=1pt,
caption={[Multi-beam restfreq syntax example 1]},label={lst:appendix_example1}]
{appendix_example1.py}

\item {\bf Example 2}: For simple configurations the syntax for the existing
receivers would also be supported.  This results in the routing of 4 beams,
2 polarizations with each tuned to a rest frequency of 24000~MHz.

\lstinputlisting[language=PythonAstrid,captionpos=b,frame=single,framerule=1pt,
caption={[Multi-beam restfreq syntax example 2]},label={lst:appendix_example2}]
{appendix_example2.py}

\item {\bf Example 3}: Comparison of two configtool inputs where {\bf restfreq} is a list,
and input with the dictionary syntax.  The two configurations are equal.

\begin{minipage}{0.375\linewidth}
\lstinputlisting[language=PythonAstrid,captionpos=b,frame=single,framerule=1pt,
caption={[Multi-beam restfreq syntax example 3]},label={lst:appendix_example3a}]
{appendix_example3a.py}
\end{minipage}
\hfill
\begin{minipage}{0.57\linewidth}
\lstinputlisting[language=PythonAstrid,captionpos=b,frame=single,framerule=1pt,
caption={[Multi-beam restfreq syntax example 4]},label={lst:appendix_example3b}]
{appendix_example3b.py}
\end{minipage}

\item {\bf Example 4}: 8 different rest frequencies specified.
\lstinputlisting[language=PythonAstrid,captionpos=b,frame=single,framerule=1pt,
caption={[Multi-beam restfreq syntax example 5]},label={lst:appendix_example4}]
{appendix_example4.py}

\item {\bf Example 5}: A configuration that specifies delta frequencies.
\lstinputlisting[language=PythonAstrid,captionpos=b,frame=single,framerule=1pt,
caption={[Multi-beam restfreq syntax example 6]},label={lst:appendix_example5}]
{appendix_example5.py}

\end{itemize}