\chapter{VLBI Observing using the GBT}\label{chap:vlba}

\section{Proposals}
The \gls{GBT} has a \gls{VLBA}-compatible data acquisition system.  
Proposals requesting \gls{GBT} participation in \gls{VLBA} or global \gls{VLBI} 
observations should be submitted to the \gls{VLBA} only, not to the \gls{GBT}.

Proposals requesting the \gls{GBT} participation in a \gls{VLB} experiment that
includes no other \gls{NRAO} telescopes should be submitted to the
\gls{VLBA} as well as to the \gls{GBT} and other agencies as appropriate, such
as the \gls{EVN}.

References for \gls{VLBA} proposals: \htmladdnormallink
{https://science.nrao.edu/facilities/vlba/proposing}
{https://science.nrao.edu/facilities/vlba/proposing}

General information about the VLBA: \htmladdnormallink
{https://science.nrao.edu/facilities/vlba/}
{https://science.nrao.edu/facilities/vlba/}

\section{VLBA-compatible recording}
The data acquisition system is similar to those at the \gls{VLBA} stations:
two \gls{RDBE} units and a Mark5C recorder are in use, allowing wide-band recording
up to 2 Gbits/sec. Two modes are available, \dq{PFB} mode provides 16 32-MHz
channels and a total  recording rate of 2 Gbits/sec.  The \dq{DDC} mode allows up
to 4 channels of bandwidth 1 to 128~MHz. With two \glspl{RDBE} available, up to
8 \gls{DDC} channels may be used.

 The SCHED default frequency setups should be correct for writing schedules
for the new system.

The old data acquisition system with the DAR rack and Mark5A recorder has been 
retired.  No proposals should request it.

\newpage

\section{Schedule Preparation}
Scheduling is done through the \gls{VLBA} analysts in Socorro.  
\begin{itemize}
\item Schedules are prepared with the SCHED program.   \newline
(refer to: \htmladdnormallink
{http://www.aoc.nrao.edu/software/sched/index.html}
{http://www.aoc.nrao.edu/software/sched/index.html})

\item The \gls{GBT} uses the standard \gls{VLBA} schedule files
({\tt*.key} and {\tt*.vex} files).  
\item The user needs to prepare a {\tt.key} file for SCHED and send it
to the \gls{VLBA} analysts.

\item Use {\bf GBT\_VLBA} as the station name, except for cold weather
in which case use {\bf GBT\_COLD}. Refer to pointing and weather sections, below.  
\item In general, use {\bf GBT\_COLD} during the months of December, January,
and February.
\end{itemize}

The schedules, either {\tt.vex} or {\tt.key} files, are processed by the \gls{VLBA}
analysts to produce schedule scripts for each \gls{VLBA} telescope, including one
for the \gls{GBT}.  These scripts are interpreted at the \gls{GBT} by a process
called \dq{RunVLBI} which generates the configuration and pointing commands for
the \gls{GBT}. The same script runs in the \gls{VLBA} backend to drive the
recording and backend frequency setup.  The \gls{GBT} telescope operator runs these
experiments. The user does not need to know anything about GBT-specific script
details, i.e, the \gls{Astrid} configurations, catalogs, and scheduling blocks.

There are, however,  several GBT-specific details which the user needs to take
into consideration when designing the observing schedule.  These are described in
the next few sections.

\section{Special considerations when using the GBT}

\begin{itemize}
\item Allow about 30 minutes setup time at the beginning of a session before
\gls{VLBI} recording begins, except for the 3mm and 7mm receivers for which one
hour setup time should be allowed.
 
\item Changing between Gregorian receivers requires rotating the turret.
      The telescope operator initiates this rotation.  At least 5 minutes
      should be allowed in the schedule to change from one Gregorian receiver to
      another.
        
\item  Changing between Gregorian and prime focus requires about 10 minutes;
       that is the time required to extend or retract the prime focus boom.  
       Changing from one prime focus receiver to another requires about 4
       hours, because one feed must be physically removed and replaced with
       another.

\item The prime focus receivers include 50 cm and 90 cm bands; 
      whereas \gls{Lband} and all higher frequencies ($\nu > $ 1.2~GHz)
      use the Gregorian focus.

\item One needs to include enough pointing/focus updates -- see below.
  
\item There are some weather-related restrictions -- see below.
  
\end{itemize}

\newpage

\section{Available Receivers and Bands}
The receivers and frequency bands are listed in table~\ref{table:vlbabands}. 
Note that some bands are available on the \gls{GBT} but not on the \gls{VLBA}.
Note also the time it takes to change receivers, as described above. For more
information, consult
\begin{itemize}
\item The GBT proposers guide, chapter 4, for antenna and receiver performance.
\item Gain curves, see \htmladdnormallink
{https://safe.nrao.edu/wiki/bin/view/GB/Observing/GainPerformance}
{https://safe.nrao.edu/wiki/bin/view/GB/Observing/GainPerformance}
\end{itemize}

\begin{table}[!h]
\begin{center}
\caption[VLBA bands and GBT receivers]{VLBA bands and GBT receivers.
\label{table:vlbabands}}
\begin{tabular}{c c c c c c c c }
\toprule

VLBA & VLBA      & GBT       & GBT      & Net  & Primary & Est. & Typical \\
Band & Frequency & Frequency & Receiver & Side-& Beam    & SEFD & Tsys    \\
     & Range     & Range     &          & band & FWHM    & (Jy) & (K)     \\
     & (GHz)     & (GHz)     &          &      &         &      &         \\

\midrule
90 cm & 0.312-0.342 & 0.290-0.395 & Rcvr\_342 (PF1) & lower & 36\arcminute & 25 & 20-70 \\
--- & --- & 0.385-0.520 & Rcvr\_450 (PF1)  & lower & 27\arcminute & 22 & 20-50 \\
50 cm & 0.596-0.626 & 0.510-0.690 & Rcvr\_600 (PF1) & lower & 21\arcminute & 12 & 20-35 \\
--- & --- & 0.680-0.920 & Rcvr\_800 (PF1) & lower & 15\arcminute & 15 & 18-25 \\
--- & --- & 0.910-1.230 & Rcvr\_1070 (PF2) & lower & 12\arcminute & 10 & 18-22 \\
\midrule
18/21 cm & 1.35-1.75 & 1.1-1.8 & Rcvr1\_2 & lower & 9\arcminute & 10 & 15-18 \\
13 cm & 2.15-2.35 & 1.68-2.60 & Rcvr2\_3 & lower & 5.8\arcminute & 12 & 18 \\
6 cm & 3.9-7.9 & 3.95-8.0 & Rcvr4\_6 & lower & 2.5\arcminute & 10 & 23 \\
4 cm & 8.0-8.8 & 7.9-10.1 & Rcvr8\_10 & lower & 1.4\arcminute & 15 & 27 \\
2 cm & 12.0-15.4 & 11.8-18.0 & Rcvr12\_18 & upper & 54\arcsecond & 20 & 27 \\
\midrule
1 cm & 21.7-24.1 & 18.0-27.5 & RcvrArray18\_26 & lower & 32\arcsecond & 25 & 40 \\
--- & --- & 26.0-40.0 & Rcvr26\_40 & upper & 22\arcsecond & 20-40 & 40 \\
7 mm & 41.0-45.0 & 40.0-50.0 & Rcvr40\_52 & upper & 16\arcsecond & 60 & 80 \\
3 mm & 80.0-90.0 & 68-92 & Rcvr68\_92 & upper & 10\arcsecond & 100 & 110 \\
\bottomrule
\end{tabular}
\end{center}
\end{table}

\noindent\textbf{Notes:}
\begin{itemize}
\item Receivers with \dq{PF1} or \dq{PF2} are at the prime focus; the others
are at the Gregorian focus.
\item Rcvr26\_40 has linear polarization only; 2 beams but one polarization
state per beam; all other receivers can receive dual circular polarizations.
\item Pulse Cal (or phase cal) is injected in receivers of 2 cm wavelength and
longer; pulse cal is injected in the 7mm receiver after the first mix; other
receivers have no pulse cal injection.
\item The 3mm receiver (Rcvr68\_92) has no noise cal or pulse cal injection.
See the section below for how calibration is done.
\end{itemize}


\newpage

\section{Include Pointing and Focus Checks}

It is recommended to allow for pointing and focus touch-ups when observing at
the higher frequencies. Recommendations are listed in
table~\ref{table:vlbapointing}.

\begin{table}[!h]
\begin{center}
\begin{tabular}{cc}
\toprule
Frequency Band &  Interval between pointing scans \\
\midrule
4--10~GHz & 4--5 hours \\
12--18~GHz & 3--4 hours \\
18--26~GHz & 1.5--2 hours \\
40--90~GHz & 30--60 minutes \\
\bottomrule
\end{tabular}
\caption[GBT pointing and focus checks with VLBA observations]
{GBT pointing and focus checks with VLBA observations.
\label{table:vlbapointing}}
\end{center}
\end{table}

\noindent\textbf{Notes:}
\begin{itemize}
\item The observer should select a strong continuum source (flux density
$>$ 0.5 Jy, or $>$ 1.0 Jy for $\nu >$ 20~GHz).
\item Allow about 6 minutes for the pointing/focus check, except for the 3mm
receiver for which you should allow 8 minutes in order to include the
temperature calibration.
\item For observing at frequencies below 5~GHz, include one pointing scan at
the beginning of the session.
\item The telescope operator will usually do a point/focus scan at the beginning
of an observing session, during the startup time.
\end{itemize}

To include a point/focus scan in your schedule, put commands into your {\tt.key}
file similar to the following:

\lstinputlisting[captionpos=b,keepspaces=true,basicstyle=\ttfamily,
frame=single,framerule=1pt,backgroundcolor=\color{white},
caption={[pointing and focusing with the GBT]Example additions to the {\tt.key}
file for GBT point and focus observations},
label={lst:vlba_pointing}]
{vlba_pointing.key}

It is important to specify only the \gls{GBT} ({\tt stations=gbt\_vlba} or
{\tt stations=gbt\_cold}) when putting in {\tt peak=1}.  Otherwise it may do
a reference pointing for the whole \gls{VLBA}, and if the pointing source is
under about 5 Jy, it can produce bad results. Refer to the SCHED manual for
details of schedule preparation at \htmladdnormallink
{http://www.aoc.nrao.edu/software/sched/index.html}
{http://www.aoc.nrao.edu/software/sched/index.html}

\subsection{3mm Receiver (68-92 GHz) calibration}
System Temperature ($T_{sys}$) calibration with this receiver uses a calibration
wheel that can place hot and cold loads in front of the feed. There is no noise
injection as happens with the other receivers. A \dq{cal sequence} procedure is
done before and after each peak/focus to provide a $T_{sys}$ measurement. A cal
sequence is inserted automatically with the peak/focus; the user does not have
to specify it explicitly. A cal sequence takes about one minute, and will happen
before and after a peak/focus. The user should use a dwell time of 8 minutes
for the pointing scan, and that will include the cal sequences. Pointing Sources
for high frequency observing should be strong, i.e., stronger than 3 Jy if possible.

\newpage

\section{Weather Considerations}
At the higher frequencies, windy conditions can degrade the pointing.
Refer to recommended wind limits for observing at \htmladdnormallink
{https://safe.nrao.edu/wiki/bin/view/GB/PTCS/PointingFocusGeneralStrategy}
{https://safe.nrao.edu/wiki/bin/view/GB/PTCS/PointingFocusGeneralStrategy}

\begin{itemize}
\item For sustained winds of $>$ 35 MPH or gusts $>$ 40 MPH, the telescope
is stowed for safety.
\item Ambient temperature $<$ 17F (-8.3\celsius) : the maximum azimuth slew rate
is reduced to $18^{\circ}$/min.
\item Ambient temperature $<$ -10F (-23\celsius) : the antenna is shut down.
\end{itemize}

If your project will run in December, January, or February you should use the
lower azimuth slew rate of $18^{\circ}$/min when making the schedule. This is
accomplished by using {\tt stations=gbt\_cold} in your {\tt.key} file, instead
of {\tt stations=gbt\_vlba}.

\section{Telescope Move times and limits}
\begin{description}
\item[Move Limits:]\ 
\begin{itemize}
\item Elevation: $5^{\circ} \rightarrow 90^{\circ} $
\item Azimuth: $-90^{\circ} \rightarrow +450^{\circ}$,
i.e, $180^{\circ} \pm 270^{\circ} $
\end{itemize}
\item[Calculating time to change sources:]\ 
\begin{itemize}
\item {\bf Maximum Azimuth slew rate}: $36^{\circ}$/min ($17^{\circ}$/min at low
temperature)
\item {\bf Maximum Elevation slew rate}: $18^{\circ}$/min
\item {\bf Acceleration}: $0.05^{\circ}$sec$^{-2}$
\item {\bf Overhead}: 20 seconds to settle
\item Allow a minimum of 30 seconds for a source change, even for short moves.
\end{itemize}
\end{description}

\section[High Frequency (40-90 GHz) active surface considerations]
{High Frequency (40-90 GHz) active\\ surface considerations}

When using the 40-50 or 68-92 GHz receivers, one should tune up the active
surface by doing an \dq{AutoOOF} procedure. This is so-called \dq{Out of focus
holography} in which a strong point source is observed both in and out of
focus, and large-scale deviations of the surface can be derived. The surface
corrections are applied to the active surface model. This improves the
aperture efficiency by a factor of 2 at 86 GHz. One should do an AutoOOF,
which takes about 30 minutes, at the beginning of any high-frequency
observing. The user does not have to specify this in the observing file;
the operator or telescope friend will do an AutoOOF calibration prior to
starting the observing, during the setup.

When observing with the 68-92 GHz receiver, one should repeat the AutoOOF
about every 3-4 hours. This means that the user should allow a 30 minute gap
in the schedule about every 3-4 hours. The user does not have to specify
anything about an autoOOF in the schedule; just allow the 30 minute gap.
The operator or telescope friend will do the calibration. 

\section{GBT Coordinates}

The geodetic position for the GBT (as of Jan 2000), based on a local survey
referred to a standard NGS survey marker on the Green Bank site in the NAD83
system is
\begin{itemize}
\item longitude = $79^{\circ} 50\arcminute 23.406\arcsecond W$
\item latitude = $38^{\circ} 25\arcminute 59.236\arcsecond N$
\item Height of Track: NAVD88 height: 807.43 m (wrt ellipsoid: 776.34 m)
\item Height of elevation axle: NAVD88 height: 855.65 m (wrt ellipsoid: 824.55 m) 
\end{itemize}

The surveyed height refers to the top of the azimuth track. The phase center
(intersection of azimuth and elevation axes) is 48.22m above the top of the
azimuth track. The average geoid height = -31.10m with respect to the
ellipsoid. The estimate uncertainty is 0.04\arcsecond.

The Earth-centered \gls{ITRF} coordinates for the phase center of the \gls{GBT}
were derived from a \dq{TIES} run with the \gls{GBT} and 20-meter telescopes in
December 2002.  Geodetic solution for the \gls{ITRF} coordinates may be found
through the web site:\\
\htmladdnormallink
{http://gemini.gsfc.nasa.gov/solutions/}
{http://gemini.gsfc.nasa.gov/solutions/}

\noindent The solution as of Oct 2007 is:
\begin{align*}
x =& 882589.638\ meters \\
y =& -4924872.319\ meters \\
z =& 3943729.355\ meters
\end{align*}

\noindent Based on the \gls{ITRF} solution, the best NAD83 geodetic position is:

\begin{align*}
\text{Latitude} &= 38^{\circ} 25\arcminute 59.266\arcsecond N\ (38.433129^{\circ} N)  \\
\text{Longitude}&= 79^{\circ} 50\arcminute 23.423\arcsecond W\ (79.839840^{\circ} W)  \\ 
\text{Height above the ellipsoid}&= 824.36\ m \\
\text{Height above the geoid}&= 855.46\ m 
\end{align*}