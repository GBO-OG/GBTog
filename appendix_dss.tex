\chapter{DSS Control Parameters}\label{appendix:dsscontrolparameters}


\begin{itemize}
\item {\bf Required Minimum Duration}: Minimum time for scheduling a session
\item {\bf Required Maximum Duration}: Maximum time for scheduling a session
\item {\bf Time between sessions}: Time which must elapse between the end of one scheduled session period 
and the start of the next, typically used to allow the observer to sleep or reduce data.
\item {\bf  Minimum effective system temperature ($\xi$)}: 
Some observers may wish to have their project scheduled even
if the weather is not ideal.  For example, projects that use very short integration times are dominated by
overheads, not the radiometer equation, and so they may wish to get a scoring boost in order to allow observing
under a wider range of weather conditions.  To implement this desire, we allow observers to modify the
minimum effective system temperature, $T_{sys}^{'}e^{\tau^{'}}$.  This value usually has a default
set by the DSS, and depends on observing frequency.  Here, T$_{sys}$ is the atmospheric system temperature 
and $\tau$ is the opacity. Recall the atmospheric observing efficiency is
$\eta_{atm} = ({{T_{sys}^{'}e^{\tau'}}\over{T_{sys}e^{\tau}}})^2$. 
(The prime denotes the minimum value).  The minimum effective system temperature can be scaled by a factor $\xi$ to either 
improve or degrade the atmospheric conditions for a particular session. The default is $\xi = 1.0$. The user needs to use caution
when modifying this parameter, especially because it will have a different effect at different frequencies.
For example, doubling the minimum effective system temperature will have a much larger effect at Ku-band than at K-band.

The parameter $\xi$ enters into the scoring in two ways.  It affects the computation of overall observing efficiency,
$\eta_{total} = \eta_{atm} \times \eta_{tracking} \times \eta_{surface}$ and it also enters by modifying the threshold, minimum value.
\item {\bf Tracking Error Threshold, source size, and Tracking Efficiency:}
To control the effect of the expected tracking error on scheduling a session, the observer should be able to 
modify either of two values:
\begin{itemize}
\item{f$_{max}$: the tracking error limit in units of HPBW
(default 0.22 for Rcvr68\_92, 0.4 for Rcvr\_PAR and 0.2 for all other receivers)}
\item{$\theta_{src}$: the nominal source size in units of arc seconds (default 0.0)}
\end{itemize}

The tracking error is called $f$, and the tracking error limit is called f$_{max}$. If 
$f > f_{max}$ the observation is too inefficient and does not get scheduled. 
Keep in mind that the tracking error $f$ comes into play not only in regard to the 
limit, but also in the scoring equation.  The value of $f$ is ultimately a function of wind 
speed and observing frequency: $$f = \frac{\sigma}{\theta_b}$$
where
$$\left(\frac{\sigma}{arcsec}\right) \simeq \sqrt{\sigma_0^2+\left(\frac{|v|}{3.5\,m\,s^{-1}}\right)^4}
\;\;\;\;\;\;(DSPN\;18.1\;\;equation\;1)$$ 
and $$\left(\frac{\theta_b}{arcsec}\right) \simeq \frac{748}{\nu}$$
with $\theta_b$ being the HPBW, $\sigma_0$ being the rms tracking error in the absence of wind, equal to 
1.32\arcsecond\ at night and 2.19\arcsecond\ during the day, and $\nu$ being the observing frequency
in GHz.

A value $f_{max}$ = 0.2 assures observers that their flux uncertainty due to 
tracking errors will be no more than 10\%, assuming they are observing a point 
source ($\theta_{src}$ = 0.0).  Observers who wish to do better than 10\% may decide 
to specify a smaller value of $f_{max}$.  For example $f_{max}$ = 0.14 assures no more 
than 5\% flux uncertainty due to tracking errors.

Some observers may wish to relax the tracking restrictions because their source 
is extended, not point-like.  So the most natural way for them to ease the 
constraint is to specify a source size: $\theta_{src}$.  The default value is $\theta_{src} = 0.0$
arc seconds.  If the user specifies 
$\theta_{src}$ then the tracking error $f$ should be calculated as follows: $f = \frac{\sigma}{\theta_{obs}}$
where $\theta_{obs} = \sqrt{\theta_{src}^2 + \theta_b^2}$.  This new value for the tracking error
must then be used in calculating the tracking efficiency, $\eta_{tr}$.
So changing the source size impacts the scoring equation through both the tracking efficiency
and the tracking error limit.

So to summarize, most observers will not need to modify $f_{max}$ or $\theta_{src}$. 
If they do, it would be most sensible to modify only one or the other.  If 
observing point sources, the observer may wish to change $f_{max}$ to tighten or 
loosen the requirements.  If observing extended sources, the observer should 
stick with $f_{max}$ = 0.2 and change the value of $\theta_{src}$.

Specifying both $f_{max}$ and $\theta_{src}$ need not be forbidden by the software, but 
it is probably not the best approach.
\item {\bf Irradiance}: Section 3.4.5 of DSPN5 describes the concept of irradiance, an important concept for
continuum observations above 2 GHz.  Most continuum observations prefer the default values of
irradiance (300 W M$^{-2}$), there are times when the value
should be tunable, and any value of irradiance may be used for the DSS.
\item {\bf Elevation Limit}: This parameter allows the observer to modify the frequency dependent
hard elevation limit, and instead set it to any value (in degrees).  When using this parameter, 
the hour angle scoring factor will be set to zero when the source for a session is less than the minimum allowed elevation and set to one otherwise.
\item {\bf Solar Avoidance}: The angle by which the project must avoid the sun.  The default here is 0.
\item {\bf Time of Day}: Time of day restrictions for this session.  Options are: Any Time of Day (default); RFI (8pm - 8am);
and PTCS (sunset -- sunrise+2hours).

\item {\bf Transit}:  The central coordinates of this session must pass through transit, with  at least 25\% of the session 
on either side of the transit window.
\item {\bf LST Exclude/Include:} This allows the session to exclude/include LST ranges when scheduling.  More than one range can be give, but they must be listed sequentially.
\item {\bf Keyhole Limit}: Boolean to set a maximum elevation, specified by the sessions
primary (first in list) receiver.  When set to true, sessions not requiring Mustang will not be scheduled
when their source will be above 80 degrees in elevation during the duration of the telescope
period;  Mustang observations will not be scheduled when the source will be above 78 degrees
in elevation during the duration of the telescope period.
\item {\bf Good Atmospheric Stability (GAS)}: The atmospheric stability limit, $\ell_{st}$, is a factor in the scoring algorithm 
(see DSPN5). It is used only for continuum observations which are sensitive to atmospheric fluctuations. Currently, a forecast 
downward irradiance, $I_{down}$, threshold value of 300 W/mi$^2$, is used to derive  $\ell_{st}$.  However, a different metric has been 
developed for the 90 GHz Bolometer array, MUSTANG, that uses the atmospheric system temperature (including hydrosols) at the target 
position elevation.   GAS is used to set the value of $\ell_{st}$ for MUSTANG only and is ignored for all other receivers, as follows:
\begin{itemize}
For MUSTANG {\bf only} derive the atmospheric stability limit by first calculating the zenith atmospheric system temperature at 
90 GHz.  (This includes hydrosols, the default in the CLEO command line interface, and is described in equation 7 in DSPN5.)
The atmospheric system temperature is the last term on the right hand side of the equation, 
$T_{sys}^{atm}(El = 90^\circ)\;=\;T_k(1-e^{(EL=90^\circ)})$. Then derive $T_{sys}^{atm}(El = 90^\circ)/Sin(El)$,  
the the low opacity atmospheric system temperature. For MUSTANG, assume a frequency of 90 GHz. When GAS is TRUE ("good")
then  $T_{sys}^{atm}(El = 90^\circ)/Sin(El)\;<\; 35K$ then $\ell_{st}$=1; otherwise $\ell_{st}$=0.
If GAS is FALSE ("usable") then if  $T_{sys}^{atm}(El = 90^\circ)/Sin(El)\;<\; 50K$ then $\ell_{st}$=1; otherwise $\ell_{st}$=0.
 \end{itemize}
For all other receivers derive the atmospheric stability limit as usual: if $I_{down}$ <  300 W/mi$^2$ then $\ell_{st}$=1; otherwise $\ell_{st}$=0.
\end{itemize}
