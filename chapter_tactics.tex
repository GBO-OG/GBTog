\chapter{Observing Tactics and Recommendations}\label{chap:tactics}

%============================================================================

\section{Active Surface (AS) Strategies}

If you are observing at a frequency of 8~GHz or higher then you should
use the \glsfirst{AS}.  At frequencies below 8~GHz the \gls{AS} does not provide any
improvements to the efficiency of the \gls{GBT}. Due to \gls{RFI} considerations
the \gls{AS} may be turned off for lower frequency observations.

{\bf You do not need to do anything to turn on or off the use of  the \gls{AS}.
The \gls{GBT} telescope operator performs these tasks.}

%============================================================================
\section{AutoOOF Strategy}\label{sec:oof_strategy}
AutoOOF is recommended for observing at frequencies of 26~GHz and higher and only
available for use with Rcvr26\_40 (\gls{Kaband}), Rcvr40\_52 (\gls{Qband},
Rcvr68\_92 (\gls{Wband}, and Rcvr\_PAR (\gls{MUSTANG}).
For a description of this procedure, refer to Section~\ref{sec:AutoOOFsec}.
For the associated data display, see Section~\ref{sec:oofdatadisplay}.
It is important to note the following points when using AutoOOF:

\begin{description}
\item[Choose a bright calibrator:] preferably at least 7 K in the observed band, which
is about 4 Jy at Q-band. You should not rely on the catalog flux to be accurate
as it is often many years out of date.  The ALMA Calibrator Source Catalogue has
an extensive record of the flux densities for many of the bright 3mm sources
(\htmladdnormallink{https://almascience.eso.org/sc/}{https://almascience.eso.org/sc/}).
If you are not sure then run a point/focus scan on the calibrator first in order to
confirm its strength.  Remember, you need to be able to detect the source when
the subreflector is $\pm{5}\lambda$ out of focus which typically reduces its peak
intensity by a factor of 8.

\item[Allow approximately 20 minutes for an AutoOOF:] The AutoOOF procedure will
obtain three \gls{OTF} maps, each taken at a different focus position.  Each map
will require nearly 1 minute of intial \gls{MC} overhead and last 316 seconds
with Rcvr26\_40, 300 seconds with Rcvr40\_52, and 366 seconds with Rcvr68\_92.
Processing will then require 1 minute for a total of 19--22 minutes.

\newpage

\item[Use AutoOOF to derive pointing and focus offsets:] The processing is launched
automatically upon completion of the third map, and the result is displayed in the
OOF plug-in tab of Astrid. It is incumbent upon the user to examine the solutions,
and click the button (in the Astrid DataDisplay tab) to send the selected solution
to the active surface. It is recommended that when sending the solutions, you use
the yellow button in the OOF display tab labeled \dq{Send Selected Solution with
Point and Focus Corrections}. By using this method, it is no longer necessary to
follow AutoOOF with another AutoPeakFocus.

\item[Running AutoOOF first:] If you plan to run AutoOOF as the first thing during
your observing session, we recommend running an AutoPeakFocus before the AutoOOF
for all receivers except Rcvr68\_92 (See next point). Subsequent runs of AutoOOF will
not need a pre-point/focus as small errors in these values do not harm the results.

\item[AutoOOF with the W-band (68--92~GHz) recevier:]\ \\
\begin{itemize}
\item {\bf Use AutoOOF before AutoPeakFocus:} Blind pointing offsets at W--band
(68--92~GHz) frequencies are similar to beam sizes.  The source may be missed
in the simple Az--El scans used by the Peak procedure.  In These cases, AutoOOF should
be run before AutoPeakFocus.
\item {\bf AutoOOF is not necessary for daytime observations of extended sources:}
While beam sizes can vary significantly during the day (10--14\arcsecond), this has little
effect on main-beam efficiency used for the calibration of extended sources.  Therefore,
extended sources may be observed during the day without the AutoOOF corrections if the
science is not impacted by the primary beam variations.
\end{itemize}

\item[AutoPeakFocus after AutoOOF:] AutoPeakFocus may be run as a sanity check on the
AutoOOF solution.  If Peak/Focus scans were performed before AutoOOF, then source amplitude
should be greater than what was seen before the surface correction was sent.  Additionally,
AutoPeakFocus pointing and focus corrections should agree with values derived by AutoOOF.

\end{description}

\subsection{How long does the solution remain valid?}
\begin{itemize}

\item {\bf Nighttime}: If the corrections are measured at least an hour after sunset,
then they should last for the next few hours as the backup structure cools off.
This can take many hours if it was a sunny day.  At frequencies below 90~GHz, turning
off the corrections sometime between midnight and 3AM may improve the surface if a
sidelobe begins to appear on bright pointing sources.

\item {\bf Daytime}: During the daytime, this is a difficult question to answer, as it
depends on how much the pose of the telescope is changing with respect to the Sun,
cloud cover changes, etc. The answer can be anything from 1-4 hours. In practice, we
suggest running an AutoPeak every 30-40 minutes to look for characteristics outlined below.

\item {\bf Periodically Examine Peak Scans}: A new AutoOOF may be necessary if the
following characteristics are seen:
\begin{itemize}
\item Significant sidelobes begin to appear.
\item The beam size increases by more than 10\%.
\item Source amplitude decreases systematically by 15\% or more.
\end{itemize}
\end{itemize}

\noindent More information on AutoOOF can be found at\\ \htmladdnormallink
{https://safe.nrao.edu/wiki/bin/view/GB/PTCS/AutoOOFInstructions}
{https://safe.nrao.edu/wiki/bin/view/GB/PTCS/AutoOOFInstructions}


%============================================================================
\newpage
\section{Balancing Strategies}\label{sec:balancestrategy}

The \gls{GBT} \gls{IFsys} has many ways to add gain and/or attenuation
in the \gls{IFpath}, depending upon the desired configuration. Before taking
data with the \gls{GBT}, the observer must ensure that all components along
the \gls{IFpath} have optimum input power levels; this process is referred to
as \dq{balancing}. This will ensure for example that no components saturate
and that amplifiers are in the most linear part of their dynamic range.
The system automatically adjusts power levels to optimum values when you 
issue the {\bfseries{\textcolor{pythonKeywords}{Balance}}()} command in
\gls{Astrid}. The following discussion gives guidelines for when and how often to
use the {\bfseries{\textcolor{pythonKeywords}{Balance}}()} command.

Strategies for balancing the \gls{IF} power levels depend upon the backend,
the observing frequency, the observing tactics, the weather and the objects
being observed. The \gls{DCR} has a dynamic range of about 10
\footnote{From about 0.5 to 5 Volts of \gls{IF} power in the \gls{IFRack}.}
in its ability to handle changes in the brightness of the sky as seen by the 
\gls{GBT}. The sky brightness can change because of continuum emission of a
source or a maser line as you move on and off the source.  It can also change
due to changes in the atmosphere's contribution to the system temperature
as the elevation of the observations change or due to the weather.

Whenever you {\bfseries{\textcolor{pythonKeywords}{Balance}}()} you almost
always change the variable attenuator.  Each attenuator setting has a
unique bandpass shape. So if you change attenuators then you will likely see
changes in the bandpasses and \glspl{baseline} of the raw data.  

There are not any set--in--stone rules for when an observer should
balance the \gls{GBT} \gls{IFsys}.  However there are some guidelines which
will allow you to determine when you should balance the \gls{IFsys}.  Here are
the guidelines:

\begin{enumerate}[label=\bfseries{\arabic*.},leftmargin=*,
labelindent=\parindent]
\item You should balance the \gls{IFsys} after performing a configuration.  
\item You should minimize the number of times you balance when observing.
\item If you know ${\rm \gls{Tsys} + \gls{Tsrc}}$ will change by more than
a factor of two (3~dB\footnote{A change in power from ${\rm P_1}$ 
to ${\rm P_2}$ can be represented in dB by ${10 \log_{10}{(P_1/P_2)}}$})
when you change sources (not between and on and off observation) you should
consider balancing.
\item Try to avoid balancing while making maps.
\item Never balance between \dq{signal} and \dq{reference} observations (such
as during an on/off observation).
\item If you are observing target sources and calibration sources then
try not to balance between observations of the targets and calibrators.
\end{enumerate}

\noindent {\bf Note}:
\begin{itemize}[leftmargin=*]
\item If during your observing you expect to see a change in power levels on
the sky that are roughly equivalent to the \gls{GBT} system temperature, then
you should contact your \gls{GBT} support person to discuss balancing strategies.
There are no global solutions and each specific case must be treated independently.

\item If the system temperature between \dq{signal} and \dq{reference} observations
differ by a factor of two or more, {\bfseries{\textcolor{pythonKeywords}{BalanceOnOff}}()}
(see \S~\ref{sec:balanceonoff}) should be used in place of
{\bfseries{\textcolor{pythonKeywords}{Balance}}()}. This will balance the \gls{IF}
power levels at the midpoint of the two power levels at each sky position.

\end{itemize}

\newpage

\subsection{Balancing VEGAS}

There are two aspects to \gls{VEGAS} balancing that will produce separate error
messages, so the user should check the origin carefully:

\begin{enumerate}[label=\bfseries{\arabic*.},leftmargin=*,
labelindent=\parindent]
\item {\bf Adjusting \gls{IF} power levels upstream of the \gls{VEGAS} \gls{ADC}}:
Power levels upstream of the \gls{VEGAS} \glspl{ADC} are balanced so that the
power going into an \gls{ADC} is at an acceptable level.

\begin{itemize}[leftmargin=*]
\item {\bf Balancing will fail if the input \gls{IF} power levels to \gls{VEGAS}
is more than $\mathbf{\pm 2}$ dB from the target value of -20 dBFS}. \gls{VEGAS}
has a much higher range than that, but it is extremely rare that the
observation/equipment combination should prevent the balancing algorithm from
meeting the target.  A conservative limit was chosen since a failure normally
means that some part of the system is not configured correctly, or that there
is hardware failure.  If an \gls{IF} balancing failure occurs, the user
(or operator) should look for errors in the \gls{IFsys} and can view
the actual \gls{IF} power levels in the \gls{VEGAS} \gls{CLEO} screen
(see \S~\ref{sec:vegas_cleo}).  If they are significantly different from
-20 dBFS, there is a problem somewhere in the \gls{IF} chain.

\item {\bf If the power levels are different from -20 dBFS, but close to it,
there may not be a real problem}.  In some cases, the \gls{IF} balancing will
fail due to an excceptional, but acceptable circumstance; for example, looking
at an extremely bright source, or using a spectral window close to the edge
of the recevier passband.  The \gls{IF} balancing failure does not cause an
abort, and it is often acceptable to continue observing under these circumstances.

The useable dynamic range of \gls{VEGAS} is actually $>$ 20 dB.  It is set at a
low level by quantization effects, and at high levels by saturation.  If the
\gls{IF} power level looks reasonable, the next check is to look at the
\gls{ADC} histogram counts.  As long as the histogram looks like a Gaussian
distribution, witha a FWHM around 20 counts or larger, but with no counts
approaching $\pm 127$, then the \gls{IF} level into \gls{VEGAS} is acceptable
(see Figure~\ref{fig:vegasdm}).  Make sure you monitor the \gls{ADC} histogram
through all phases of your observation (e.g. switching on and off a bright source).
\end{itemize}

\item {\bf Adjusting the \dq{digital gain} inside the \gls{VEGAS} processing
firmware}: There should be no circumstances (e.g. an FFT overflow) which result
in lost precision.
\begin{itemize}[leftmargin=*]
\item {\bf The digital gain should never fail to balance}. It is a property
of the firmware design of each mode, not the \gls{IF} input.  A failure
of the digital gain balancing indicates a serious problem, and engineering
support should be called.
\end{itemize}

\end{enumerate}

%============================================================================

\section{Calibration Strategies}

For best flux density calibration of spectra, it is recommended that you should
observe continuum flux density calibration sources at least once during an
observing session. To do this, do a Peak/Focus on the calibrator, followed
by an observation in the same spectral line setup used for the program sources.
This will give the relation of flux density to antenna temperature as a function
of frequency that can be applied to the program spectra.

Of course, there may be other outstanding reasons to perform calibration
observations more often.  If you have concerns over how often you should observe
a calibrator you should get into contact with your \gls{GBT} support person.

Flux density histories for many of the bright 3mm point sources can be found
in the ALMA Calibrator Source Catalogue
(\htmladdnormallink{https://almascience.eso.org/sc/}
{https://almascience.eso.org/sc/}).

%============================================================================
\newpage

\section{Pointing and Focusing Strategies}\label{sec:pointfocus}

How often you need to point and focus the \gls{GBT} depends on the frequency
of your observations, the weather conditions, whether or not it is day
or night-time, and the amount of flux error that your experiment can
tolerate from pointing and focus errors.  See Table~\ref{table:wind} for guidelines
on how often to Point/Focus.  Note that spacings between Point/Focus observations may
be increased if results appear stable, especially during the night.

Within the \gls{DSS}, the tracking error $\sigma_{tr}$  (arc seconds) as a function of
wind speed $s$ ($m s^{-1}$) is given by 
\begin{align}
\sigma_{tr}^2 = \sigma_0^2 + \left(\dfrac{s}{3.5}\right)^4
\label{eq:wind}
\end{align}
Where $\sigma_0=1.32$\arcsecond\ at night and $\sigma_0=2.19$\arcsecond\ during
the day, and is the tracking and pointing error with no winds. The \gls{DSS} will
only schedule observations if the tracking error is smaller than a specified fraction
($f<f_{max}$) of the beam \gls{FWHM} ($\sigma_{beam}$) given by
\begin{align}
f = \frac{\sigma_{tr}}{\sigma_{beam}} = \frac{\sigma_{tr}\nu}{748}
\label{eq:dss_wind_limit}
\end{align}

where $\nu$ is the observing frequency in GHz.  Values for $f_{max}$ in the \gls{DSS}
are currently set at 0.2 for receivers below 50~GHz, 0.22 for receivers above 50~GHz
(Rcvr68\_92) and 0.4 for filled arrays (Rcvr\_PAR).  An $f_{max}$ value of 0.2 assures
observers that their flux uncertainty due to tracking errors is no more than 10\%,
assuming they are observing a point source.

\begin{table}[!h]
\begin{center}
\caption[Observing wind default limits and Point/Focus strategies]
{Observing wind limits using DSS default parameters and suggested time periods between
Point and Focus observations.
\label{table:wind}}
\begin{tabular}{l d d d c c}
\toprule
& & \multicolumn{2}{c}{Wind limit $\left(m s^{-1}\right)$} &
\multicolumn{2}{c}{Recommended Point/Focus spacing} \\ \cmidrule(lr){3-4} \cmidrule(lr){5-6}
Receiver &
\multicolumn{1}{c}{$\nu$ (GHz)}&
\multicolumn{1}{c}{Day} &
\multicolumn{1}{c}{Night} &
\multicolumn{1}{c}{Day} &
\multicolumn{1}{c}{Night} \\
\midrule

Rcvr\_342       & 0.340  & 73.4 & 73.4 & \multicolumn{2}{c}{--- Initial Peak only ---} \\
Rcvr\_450       & 0.415  & 66.5 & 66.5 & \multicolumn{2}{c}{--- Initial Peak only ---} \\
Rcvr\_600       & 0.680  & 52.0 & 52.0 & \multicolumn{2}{c}{--- Initial Peak only ---} \\
Rcvr\_800       & 0.770  & 48.8 & 48.8 & \multicolumn{2}{c}{--- Initial Peak only ---} \\
Rcvr\_1070      & 0.970  & 43.5 & 43.5 & \multicolumn{2}{c}{--- Initial Peak only ---} \\
Rcvr1\_2        & 1.4    & 36.2 & 36.2 & \multicolumn{2}{c}{--- Initial Peak and Focus only ---} \\
Rcvr2\_3        & 2.0    & 30.3 & 30.3 & \multicolumn{2}{c}{--- Initial Peak and Focus only ---} \\
Rcvr4\_6        & 5.0    & 19.1 & 19.1 & Hourly on hot afternoons & Every 2-3 hours  \\
Rcvr8\_10       & 10.0   & 13.5 & 13.5 & Hourly on hot afternoons & Every 2-3 hours   \\
Rcvr12\_18      & 15.0   & 11.0 & 11.0 & Hourly & Every 1-2 hours \\
RcvrArray18\_26 & 25.0   & 8.3  &  8.5 & Hourly & Every 1-2 hours \\
Rcvr26\_40      & 32.0   & 7.1  &  7.4 & Hourly & Hourly \\
Rcvr40\_52      & 45.0   & 5.5  &  6.1 & Every 30-60 minutes & Hourly \\
Rcvr68\_92      & 80.0   &  -   &  4.4 & Every 30-60 minutes & Every 30-60 minutes \\
Rcvr\_PAR       & 90.0   & 5.5  &  6.1 & Every 30-60 minutes & Every 30-60 minutes \\

\bottomrule
\end{tabular}
\end{center}
\end{table}

Table~\ref{table:wind} provides wind limits using default \gls{DSS} parameters.
Observers may wish to alter some parameters in the \gls{DSS} to better suit their observing
requirements.  For example, pointing may be relaxed for extended sources (i.e.
set $\theta_{src}>0$ in the \gls{DSS}), or more tightly constrained (a value of $f_{max}=0.14$
in the \gls{DSS} assures no more than 5\% flux uncertainty due to tracking errors).
You may request changes to DSS control parameters by contacting your \gls{GBT} \dq{friend}
and emailing helpdesk-gb@nrao.edu.

For further information on \gls{DSS} control parameters see
Appendix~\ref{appendix:dsscontrolparameters}. See DSS Project Note 18.1\\
(\htmladdnormallink
{https://safe.nrao.edu/wiki/pub/GB/Dynamic/DynamicProjectNotes/dss\_20nov2014.pdf}
{https://safe.nrao.edu/wiki/pub/GB/Dynamic/DynamicProjectNotes/dss_20nov2014.pdf})
for tracking performance and parameters used in Equation~\ref{eq:wind}.

%============================================================================
\newpage

\section{Spectral Configurations}

It is good practice for the observer to schedule a short observation
towards a strong spectral line source at the beginning of each observing
session.  The observer should process and check the spectra before
proceeding with the observations to confirm the spectral configuration.
It is also important to check the system temperatures for each beam.
Poor configuration choices for the backend will sometimes be detected as
anomalous system temperatures.

%============================================================================

\section{Mapping Strategies}

The sky should be sampled five times across the beam (i.e., the scan rate and
integration times should be set such that there are five integrations recorded
in the time the telescope scans across an angle equal to the beam FWHM) and
there should be four switching periods per integration (see Mangum, Emerson,
and Greisen 2007 A\&A, 474, 679 for details).  For \gls{VEGAS}, minimum recommended
integration times and switching periods are listed in Table~\ref{tab:vegas_modes}
and \S~\ref{sec:vegas_blanking}.  The \gls{DCR} has a minimum integration time of
100 ms.

The \gls{GBT} Mapping Calculator is a useful tool for planning mapping
observations and may be used to provide \gls{Astrid} commands and
parameters for many of the mapping scan types.  The mapping calculator can be
found at \htmladdnormallink
{http://www.gb.nrao.edu/$\sim$rmaddale/GBT/GBTMappingCalculator.html}
{http://www.gb.nrao.edu/~rmaddale/GBT/GBTMappingCalculator.html}.

It is important to take account of the overheads involved with mapping scans.
For example, it takes approximately 25 seconds to start a
{\bfseries{\textcolor{pythonKeywords}{RALongMap}}} mapping scan. So observers
scheduling scans much shorter than 1 minute will lose a large fraction of their
observing time to overheads.  Using Daisy pattern scans are more efficient for
scheduling small maps, as a region can be mapped in a single scan.  However
there is an additional delay in starting a
{\bfseries{\textcolor{pythonKeywords}{Daisy}}} procedure, as the system computes
the antenna trajectories. As a rule of thumb, maps larger than about 10--20 beam
\gls{FWHM} in size should use the
{\bfseries{\textcolor{pythonKeywords}{RALongMap}}} or
{\bfseries{\textcolor{pythonKeywords}{DecLatMap}}} scan types while smaller maps
should use 
{\bfseries{\textcolor{pythonKeywords}{Daisy}}} scans. For practical reasons, it
is often best to keep the scan length under about 15 minutes.

Observers wishing to use the \gls{GBT} mapping pipeline may periodically wish
to include reference scans into their \glsdesc{SB}s.
Further information on using the pipeline can be found at\\ \htmladdnormallink
{https://safe.nrao.edu/wiki/bin/view/GB/Gbtpipeline/PipelineRelease}
{https://safe.nrao.edu/wiki/bin/view/GB/Gbtpipeline/PipelineRelease}.


%============================================================================

\section{High Frequency Observing Strategies}

When observing at frequencies above 10~GHz you should be aware that additional
calibration measurements may be necessary.  The telescope efficiency can become
elevation dependent, atmospheric opacities are important and the opacities can
be time variable.  You should contact your \gls{GBT} support person to discuss
these issues.

All the \gls{GBT} high frequency receivers have at least two beams (pixels)
on the sky.  You should make use of both of these during your observations
if possible.  For example, if you are doing position switched observations
and your source is not extended then you can use the \gls{Astrid}
{\bfseries{\textcolor{pythonKeywords}{Nod()}}} procedure to observe (see
\S~\ref{sec:observingscans}).

%============================================================================

\section[Observing Strategies For Strong Continuum Sources]
{Observing Strategies For Strong Continuum Sources}

Spectral line observations of strong continuum sources leads to a great
amount of structure (i.e. ripples) in the observed spectra. So observations of
strong continuum sources requires careful consideration of the observing
setup and the techniques used.

If you are trying to observe weak broad spectral lines (wider than
$\sim$100~MHz) toward a source with strong continuum emission (more than 1/10th
the system temperature), then you should consider using double position switching.
This technique is discussed in an Arecibo memo by Tapasi Ghosh and Chris Salter
which can be found at \htmladdnormallink
{http://www.naic.edu/~astro/aotms/performance/2001-02.ps}
{http://www.naic.edu/$\sim$astro/aotms/performance/2001-02.ps}.

Another issue is finding a proper \gls{IF} balance that allows both the \dq{on}
and \dq{off} source positions to remain in the linear range of the backend
being used.  This means that one must find the \gls{IF} balance in both the
\dq{on} and \dq{off} position and then split the difference -- assuming that
the difference in power levels between the \dq{on} and \dq{off} do not 
exceed the dynamic range of the backend.  The
{\bfseries{\textcolor{pythonKeywords}{BalanceOnOff()}}} procedure in
\gls{Astrid} can be used to accomplish this type of balancing
(see \S~\ref{sec:utilityscans}).

%===================================================================













%\section{Pulsar Observing}

%TBD.

%============================================================================
%\section{VLBI}
%
%%Scheduling is done through the VLBA analysts in Socorro. Schedules are 
%%prepared with the SCHED program. The GBT uses the standard VLBA schedule 
%%file ("*.crd" file). The user needs to prepare a ".key" file for SCHED in 
%%the usual way and send it to the VLBA analysts.
%
%The \GBT\ is different from
%the standard \VLBA\ stations.
%1) Changing between Gregorian receivers in the receiver turret takes 1-2
%minutes; and 2) Changing between Gregorian and prime focus requires 
%about 10 minutes. Changing from one prime focus receiver to another requires 
%about 4 hours, because one feed must be physically removed and replaced with 
%another. 
%
%It is recommended to allow for pointing and focus touch-ups when observing at 
%frequencies of 8\,GHz and higher. 
%Table~\ref{table:vlbi} shows the recommended maximum 
%intervals between pointings.
%At the higher frequencies (18-26 GHz and 40-50 GHz) also do a pointing check when the source elevation has changed by 15 degrees or more.
%
%
%\begin{table}[!h]
%\begin{center}
%\caption[VLBI Pointing Recommendations]{Recommended maximum intervals between
%pointing observations during VLBI observerations.\label{table:vlbi}}
%\begin{tabular}{|l|l|}
%\hline
%Frequency Band &	Interval between pointing scans \\ \hline
%8--10\,GHz  &	4--5 hours \\
%12--16\,GHz &	3--4 hours \\
%18--26\,GHz &	1.5--2 hours \\ 
%40--50\,GHz &	30--45 minutes \\
%\hline
%\end{tabular}
%\end{center}
%\end{table}
%
%The observer should select as a pointing source 
%a strong continuum source (flux density $>0.5$~Jy)
%within about 15 degrees and at similar elevation as the program source. 
%Include the pointing calibration source in the \VLBI\ observing schedule at 
%the recommended intervals. Allow about 8 minutes for the pointing and focus 
%check in the schedule.
%Note also that significant pointing errors at 7mm can happen when the wind speed is greater than 3 m/sec (7 miles per hour). For 1.3 cm significant pointing errors can occur for wind speeds greater than 6 m/sec (14 miles per hour). Refer to \S~\ref{sec:pointfocus} for details.
%
%To include a pointing and focus scan in your schedule, put commands into your 
%\dq{.key} file similar to the following:
%\begin{lstlisting}[frame=single,framerule=2pt,backgroundcolor=\color{white}]
%comment='GBT pointing scan.'  peak=1
%stations =  gbt_vlba
%source = 'J0920+4441'  dwell = 08:00  vlamode='VA' norecord /
%nopeak
%\end{lstlisting}
%It is important to specify only the \GBT\ (\dq{stations=gbt\_vlba}) when putting 
%in \dq{PEAK=1}. Otherwise it may do a reference pointing for the whole \VLBA, and 
%if the selected pointing source is under about 5 Jy for the \VLBA, it could 
%produce bad results.
%
%VLBI observers should see 
%\htmladdnormallink{http://www.gb.nrao.edu/~fghigo/gbtdoc/vlbinfo.html}{http://www.gb.nrao.edu/~fghigo/gbtdoc/vlbinfo.html}
%for more details.

%============================================================================
%\section{RADAR}

%TBD.


%============================================================================
%\section{Single Dish VLBA Recording}

%TBD.



%============================================================================
%\section{Spectral Line Observing Strategies}

%1) position switch

%2) nod

%3) frequency switch

%============================================================================
%\section{Continuum Observing Strategies}

%TBD.

%============================================================================
%\section{Balancing The Converter Rack}
%
%When using pulsar or Radar backends it may be necessary to set
%attenuation levels in the Converter Rack.  This currently does not
%happen automatically.  This can be accomplished in one of two ways.
%The first is to specify the target power level (in volts) of the
%Converter Rack modules.  This is accomplished using the BalConvRack()
%function.  Here is an example of how you would use this function in an
%Observing Script:
%
%\begin{lstlisting}
%# make the function available
%execfile("/users/tminter/astrid/BalConvRack.py")
%#
%# to use:  BalConvRack( module_list, target_values)
%#
%#   module_list is an array with a list of converter module numbers.
%#   target_values is a corresponding array with a list of RF values.
%#
%# For example if one wants to adjust CM1 to an RF value of 2.5,
%#   and CM5 to 3.5,
%#   then one says: 
%#
%BalConvRack( [1,5], [2.5, 3.5] )
%#
%#
%# This adjusts the converter module attenuators such that
%# the target power levels are attained, to within a threshhold
%# of 0.15 volts as read by the RF power samplers.
%\end{lstlisting}
%
%The second method is to specify the attenuation settings of the
%Converter Rack attenuators.  All of the Converter Rack attenuators
%can be set to the same value using:
%
%\begin{lstlisting}
%# make the function available
%execfile("/users/tminter/astrid/SetConverterRackAttenuation.py")
%#
%# set all the attenuators to 15.5 dB
%SetConverterRackAttenuation(15.5)
%\end{lstlisting}



%%%%%%%%%%%%% From balancing
%\gls{VEGAS} has the dynamic range to handle up to a factor of 10 change
%in the sky brightness.\footnote{A factor of 10 from its optimal balance
%point in each direction.}

