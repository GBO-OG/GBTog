%++++++++++++++++++++++++++++++++++++++++++++++++++++++++++++++++++++++++++++
\mychapter{Location and Offset Objects}\label{appendix:locations}

A Location is used to represent a particular location on the sky. Locations 
can be specified in the following coordinate modes: \dq{J2000}, \dq{B1950}, 
\dq{RaDecOfDate}, \dq{HaDec}, \dq{ApparentRaDec}, \dq{Galactic}, \dq{AzEl}, and \dq{Encoder.} 
A Location is specified by two values, the meanings of which are dependent on 
the coordinate mode chosen. E.g. For J2000, the two values are time and 
degrees. 

Here is an example of 
how to specify a Location:
\begin{verbatim}
location = Location("J2000", "16:30:00", "47:23:00")
\end{verbatim}

An Offset object represents a displacement from a Location.

Here is an example of 
how to specify an Offset:
\begin{verbatim}
offset   = Offset("J2000", 0.1, 0.2, cosv=True)
\end{verbatim}
which represents an offset of 0.1 degrees in RA and 0.2 degrees in DEC.
"cosv=True" (the default)  means the RA offset is divided by cosine Dec before
applying the offset.

Two Offsets may be added together, but they must have the same coordinate type.
For example:
\begin{verbatim}
off1   = Offset("J2000", 0.1, 0.2, cosv=True)
off2   = Offset("J2000", 1.0, 1.0)
totaloffset = off1 + off2

BUT:
off3   = Offset("B1950", 1.0, 1.0)
totaloffset = off1 + off3
-- result in an error!
\end{verbatim}


In Astrid scripts, one may add an Offset to a Location, if they have the same coordinate types.

Here is an example of 
how to add an Offset to a Location:
\begin{verbatim}
myoffset = Offset("B1950", "00:00:30", "00:00:45")
mysrclocation = Location("B1950", "16:30:00", "47:23:00")
mynewposition = mysrclocation+myoffset
\end{verbatim}

But note that addition is not commutative for Astrid!!
The following produces a validation error in Astrid.
\begin{verbatim}
mynewposition = myoffset + mysrclocation 
-- ERROR!
\end{verbatim}

\newpage
