\chapter{Backup Projects}\label{appendix:backupprojects}

When a scheduled telescope period is cancelled, a backup project will fill the time.
Backup projects can come in two categories: observer-run and operator-run.

\noindent
{\bf Observer-run} backup projects are those for which observers have volunteered to be called
on short notice. The notice could be as little as 15 minutes, although the GBT staff will
attempt to make the lead time as long as possible. Backup project observers should be
ready to take control of the telescope at any time of the day or night, consistent with their
observing program and blackout dates. These call-outs are expected to be rare. By
volunteering as a backup project, observers improve their project's chances of getting
observing time. Note that identifying a project as a backup does not penalize that project
during the normal scheduling procedure. The project will compete for regular scheduling
on an equal footing with all other projects, but the PI is agreeing to make the project
available as a backup in addition to regular scheduling.\\

\noindent
{\bf Operator-run}\label{sec:operatorrun} projects contain observing scripts that may be run
by the GBT operator, without need for direction from project team members. The observational
strategy must be simple. The following criteria must be met in order to be considered for
being run by the GBT operators.

\begin{description}
\item[1. ] The project must use only VEGAS or the DCR.  Other backends do not currently have
standard near real time displays which the operators can use to make sure that the data
quality appears acceptable.
\item[2. ] Minimal calibration requirements, e.g. a single pointing/focus calibration at the
beginning of the run. If the observation requires more calibration than a single
pointing/focus or simple repetition of a pointing/focus script at regular intervals
then it will not qualify as an operator-run candidate.
\item[3. ] Minimal changes in observing mode.
\item[4. ] Use of only one receiver.
\item[5. ] No scientist intervention required. An operator can be expected to determine if a
point/focus measurement is reliable but cannot be asked to judge the quality of
astronomical data. The operator also cannot be asked to judge which source
would be best to observe at any given time. If there is any doubt whether an
observation will produce reliable \dq{blind} results then this project is not suitable as
an operator-run candidate.
\item[6. ] The Astrid scripts should be as basic as possible.
\item[7. ] The operator will not edit scripts.  The PI must keep all scripts up to date.  Unfortunately,
for mapping observations this means that the maps will start over from the beginning if there
is a problem encountered while running a script since the same script will be restarted.
\item[8. ] The operators will not reduce any of the data in gbtidl or using any other data reduction package.
\item[9. ] The project will be charged for the observing time even if the data quality is not acceptable.
\item[10.] At least one telescope period for the project must have been succesfully run by the observer.
\item[11.] The PI must provide very explicit and concise instructions for the operators to follow and which
scripts to run. These must also include an example of what all the data should look like in
the Astrid Data Display tab and VEGAS display monitor.  Instructions may be stored in the 
\dq{Project Notes} on the DSS web page.

\end{description}

These requirements bias operator-run projects to low frequency observations, but high
frequency projects can be considered as well. There is no intention to implement
"service observing" by GBT scientific staff. Green Bank scientific staff will not be on
hand to check operator-run projects.

Getting a project approved as an operator-run backup requires consent from the GBT
Friend and the GBT DSS staff. To identify your project as a backup project of either
sort, inform your GBT Friend.
