\setacronymstyle{long-short-desc}

\newglossaryentry{ADC}
{
  type=\acronymtype,
  name={ADC},
  description={Analog to Digital Converter. A card used to convert an analog signal into a
  quantized digital signal.  Each \gls{VEGAS} Bank contains two ADC
  cards, one for each polarization},
  first={Analog to Digital Converter (ADC)},
} 

\newglossaryentry{API}
{
  type=\acronymtype,
  name={API},
  description={Application Programming Interface. A set of routines, protocols
  and tools that can be used when building software and applications for a
  specific system},
  first={Application Programming Interface (API)},
  firstplural={Application Programming Interfaces (APIs)}
} 

\newglossaryentry{AS}
{
  type=\acronymtype,
  name={AS},
  description={Active Surface. The surface panels on the GBT whose corner heights can be 
               adjusted to form the best possible parabolic surface},
  first={Active Surface (AS)},
} 

\newglossaryentry{Astrid}
{
  type=\acronymtype,
  name={Astrid},
  description={Astronomer's Integrated Desktop. The software tool used for executing
observations with the GBT},
  first={The Astronomer's Integrated Desktop (Astrid)}
} 

\newglossaryentry{GBT}
{
  type=\acronymtype,
  name={GBT},
  description={Green Bank Telescope},
  first={Green Bank Telescope (GBT)}
} 

\newglossaryentry{IF}
{
  type=\acronymtype,
  name={IF},
  description={Intermediate Frequency.
A frequency to which the Radio Frequency is shifted as an 
intermediate step before detection in the backend.  Obtained from mixing the 
RF signal with an LO signal},
  first={Intermediate Frequency (IF)}
} 

\newglossaryentry{IFpath}
{
  type=\acronymtype,
  name={IF path},
  description={Intermediate Frequency path.
The actual signal path between the receiver and the backend
through the IF system},
  first={Intermediate Frequency path (IF path)}
  firstplural={Intermediate Frequency paths (IF paths)}
} 

\newglossaryentry{IFsys}
{
  type=\acronymtype,
  name={IF system},
  description={Intermediate Frequency system.
A general name for all the electronics between the receiver
and the backend.  These electronics typically operate using an Intermediate
Frequency (IF)},
  first={Intermediate Frequency system (IF system)}
} 

\newglossaryentry{GBTIDL}
{
  type=\acronymtype,
  name={GBTIDL},
  description={Green Bank Telescope Interactive Data Language.
The GBT data reduction package written in \gls{IDL} for
analyzing GBT spectral line data}
} 

\newglossaryentry{AD}
{
  type=\acronymtype,
  name={A/D},
  description={Analog to Digital.
A term used for the conversion of an analog signal into 
a quantized digital signal},
  first={Analog to Digital (A/D)}} 

\newglossaryentry{CCB}
{
  type=\acronymtype,
  name={CCB},
  description={Caltech Continuum Backend.
A wideband continuum backend designed for use with the GBT 
Ka-band receiver},
  first={Caltech Continuum Backend (CCB)},
} 


\newglossaryentry{CLEO}
{
  type=\acronymtype,
  name={CLEO},
  description={Control Library for Engineers and Operators.
A suite of utilities for monitoring and controlling the GBT
hardware systems},
  first={Control Library for Engineers and Operators (CLEO)},
} 

\newglossaryentry{DCR}
{
  type=\acronymtype,
  name={DCR},
  description={The Digital Continuum Receiver. A continuum backend designed
               for use with any of the GBT receivers},
  first={The Digital Continuum Recevier (DCR)},
} 

\newglossaryentry{DDC}
{
  type=\acronymtype,
  name={DDC},
  description={Digital Down Converter.
Converts a digitized real IF signal to a complex baseband signal},
  first={Digital Down Converter (DDC)},
} 

\newglossaryentry{DSS}
{
  type=\acronymtype,
  name={DSS},
  description={Dynamic Scheduling System.
The DSS examines the weather forecast, equipment
               availability, observer availability, and other factors
               in order to generate an observing schedule},
  first={Dynamic Scheduling System (DSS)},
}

\newglossaryentry{FEM}
{
  type=\acronymtype,
  name={FEM},
  description={Finite Element Model.
This is a model for how the GBT support structure changes 
shape due to gravitational forces at different elevation angles},
  first={Finite Element Model (FEM)}
}

\newglossaryentry{FET}
{
  type=\acronymtype,
  name={FET},
  description={Field Effect Transistor.
A type of amplifier used in the receivers},
  first={Field Effect Transistor (FET)}
}

\newglossaryentry{FFT}
{
  type=\acronymtype,
  name={FFT},
  description={Fast Fourier Transform.
An approximation for a Fourier Transform which is computationally 
fast},
  first={Fast Fourier Transform (FFT)}
}

\newglossaryentry{FRM}
{
  type=\acronymtype,
  name={FRM},
  description={Focus Rotation Mount.
A mount that holds the Prime Focus Receivers which allows the 
receivers to be moved and rotated relative to the focal point.  The FRM has 
three degrees of freedom, Z-axis radial focus, Y-axis translation (in the 
direction of the dish plane of symmetry), and rotation},
  first={Focus Rotation Mount (FRM)}
}

\newglossaryentry{FWHM}
{
  type=\acronymtype,
  name={FWHM},
  description={Full Width at Half the Maximum.
Used as a measure for the width of a Gaussian},
  first={Full Width at Half the Maximum (FWHM)}
}

\newglossaryentry{FPGA}
{
  type=\acronymtype,
  name={FPGA},
  description={Field-Programmable Gate Array.
An integrated circuit designed to be programmed in the field
after manufacture},
  first={Field-Programmable Gate Array (FPGA)},
} 

\newglossaryentry{GFM}
{
  type=\acronymtype,
  name={GFM},
  description={GBT Fits Monitor: The software program that provides a real
               time display for GBT data},
  first={The GBT Fits Monitor (GFM)},
} 

\newglossaryentry{GUI}
{
  type=\acronymtype,
  name={GUI},
  description={Graphical User Interface},
  first={Graphical User Interface (GUI)},
} 

\newglossaryentry{GUPPI}
{
  type=\acronymtype,
  name={GUPPI},
  description={The Green Bank Ultimate Pulsar Procussing Instrument.
      An FPGA + GPU backend for use with GBT pulsar observations},
  first={The Green Bank Ultimate Pulsar Processing Instrument (GUPPI)},
} 

\newglossaryentry{ITRF}
{
  type=\acronymtype,
  name={ITRF},
  description={International Terrestrial Reference Frame.
A world spatial reference system co-rotating with the
Earth in its diurnal motion in space},
  first={International Terrestrial Reference Frame (ITRF)},
} 

\newglossaryentry{JD}
{
  type=\acronymtype,
  name={JD},
  description={Julian Date. A continuous count of days since the beginning
               of the Julian period (12h Jan 1, 4713 BC)},
  first={Julian Date (JD)},
} 

\newglossaryentry{KFPA}
{
  type=\acronymtype,
  name={KFPA},
  description={The K-band Focal Plane Array recevier covering 18-26.5~GHz},
  first={K-band Focal Plane Array (KFPA)},
} 

\newglossaryentry{LO}
{
  type=\acronymtype,
  name={LO},
  description={Local Oscillator.
A generator of a stable, constant frequency, radio signal used 
as a reference for determining which radio frequency to observe},
  first={Local Oscillator (LO)},
}

\newglossaryentry{LOone}
{
  type=\acronymtype,
  name={LO1},
  description={The first LO in the GBT IF system.  This LO is used to convert
  the RF signal detected by the receiver into the IF sent through the 
  electronics to the backend.  This is also the LO used for Doppler tracking},
  first={The First LO (LO1)}
} 

\newglossaryentry{LOtwo}
{
  type=\acronymtype,
  name={LO2},
  description={Second LO.
The second LO in the GBT IF system.  This is actually a set of
eight different LOs that can be used to observe up to eight different spectral
windows at the same time},
  first={Second LO (LO2)},
}

\newglossaryentry{LOthree}
{
  type=\acronymtype,
  name={LO3},
  description={Third LO.
The third LO in the GBT IF system which operates at a fixed 
frequency of 10.5\,MHz},
  first={Third LO (LO3)},
}

\newglossaryentry{LPC}
{
  type=\acronymtype,
  name={LPC},
  description={Local Pointing Correction.
Corrections for the general telescope pointing model that are 
measured by the observer},
  first={Local Pointing Correction (LPC)},
  firstplural={Local Pointing Corrections (LPCs)},
}

\newglossaryentry{LFC}
{
  type=\acronymtype,
  name={LFC},
  description={Local Focus Correction.
Corrections for the general telescope focus model that are 
               measured by the observer},
  first={Local Focus Correction (LFC)},
  firstplural={Local Focus Corrections (LFCs)}
} 

\newglossaryentry{LST}
{
  type=\acronymtype,
  name={LST},
  description={Local Sidereal Time.  A time scale based on the Earth's rate
of rotation measured relative to the fixed stars rather than the Sun},
  first={Local Sidereal Time (LST)},
} 

\newglossaryentry{MC}
{
  type=\acronymtype,
  name={M\&C},
  description={Monitor and Control.
The group of software programs which control the hardware 
               devices which comprise the GBT},
  first={Monitor and Control (M\&C)},
} 

\newglossaryentry{MJD}
{
  type=\acronymtype,
  name={MJD},
  description={Modified Julian Date.  MJD = \gls{JD} - 2400000.5},
  first={Modified Julian Date (MJD)},
} 

\newglossaryentry{RF}
{
  type=\acronymtype,
  name={RF},
  description={Radio Frequency.
The frequency of the incoming radiation detected by the GBT},
  first={Radio Frequency (RF)},
}

\newglossaryentry{NRAO}
{
  type=\acronymtype,
  name={NRAO},
  description={National Radio Astronomy Observatory.
The organization that operates the GBT, VLA, VLBA and 
the North American part of ALMA},
  first={National Radio Astronomy Observatory (NRAO)},
}

\newglossaryentry{NRQZ}
{
  type=\acronymtype,
  name={NRQZ},
  description={National Radio Quite Zone.
An area ($\sim$34,000 km$^2$) around the GBT set up by the
U.S. government to provide protection from RFI},
  first={National Radio Quite Zone (NRQZ)},
}

\newglossaryentry{OMT}
{
  type=\acronymtype,
  name={OMT},
  description={Ortho-Mode Transducer.
This is part of the receiver that takes the input from the 
wave-guide and separtes the two polarizations to go to separate detectors},
  first={Ortho-Mode Transducer (OMT)},
}

\newglossaryentry{OOF}
{
  type=\acronymtype,
  name={OOF},
  description={Out-Of-Focus holography.
A technique for measuring large-scale errors in the shape of the
reflecting surface by mapping a strong point source both in and out of focus},
  first={Out-Of-Focus holography (OOF)},
}

\newglossaryentry{OTF}
{
  type=\acronymtype,
  name={OTF},
  description={On-The-Fly.
On-The-Fly mapping scans take data while the telescope pointing
moves between two points on the sky.  This move is usually done in a linear
fashion with constant slewing speed with respect to the sky},
  first={On-The-Fly (OTF)},
}

\newglossaryentry{PI}
{
  type=\acronymtype,
  name={PI},
  description={Principle Investigator},
  first={Principle Investigator (PI)},
} 

\newglossaryentry{PFS}
{
  type=\acronymtype,
  name={PFS},
  description={A radar data acquisition backend},
  first={Portable Fast Sampler (PFS)},
} 

\newglossaryentry{RFI}
{
  type=\acronymtype,
  name={RFI},
  description={Radio Frequency Interference.
Light polution at radio wavelengths},
  first={Radio Frequency Interference (RFI)},
} 

\newglossaryentry{SB}
{
  type=\acronymtype,
  name={SB},
  description={Scheduling Block. A Python script used to perform astronomical
               observations with the GBT},
  user1={Scheduling Block},
  user2={Scheduling Blocks},
  first={Scheduling Block (SB)},
  firstplural={Scheduling Blocks (SBs)}
} 

\newglossaryentry{TAC}
{
  type=\acronymtype,
  name={TAC},
  description={Telescope Allocation Committee.
The group that decides how much observing time your proposal 
will get},
  first={Telescope Allocation Committee (TAC)},
}

\newglossaryentry{EVN}
{
  type=\acronymtype,
  name={EVN},
  description={European VLBI Network.
A collaboration of the major radio astronomical institutes
in Europe, Asia and South Africa},
  first={European VLBI Network (EVN)},
} 

\newglossaryentry{VEGAS}
{
  type=\acronymtype,
  name={VEGAS},
  description={The GBT spectral line backend},
  first={The VErsatile GBT Astronomical Spectrometer (VEGAS)},
} 

\newglossaryentry{VLB}
{
  type=\acronymtype,
  name={VLB},
  description={Very Long Baseline: A general acronym for VLBI or VLBA},
  first={Very Long Baseline (VLB)},
} 

\newglossaryentry{VLBA}
{
  type=\acronymtype,
  name={VLBA},
  description={Very Long Baseline Array: An interferometer run by the NRAO},
  first={Very Long Baseline Array (VLBA)},
} 

\newglossaryentry{VLBI}
{
  type=\acronymtype,
  name={VLBI},
  description={Very Long Baseline Interferometer: The use of unconnected
               telescopes to form an effective telescope with the size of
               the separation between the elements of the inteferometer},
  first={Very Long Baseline Interferometer (VLBI)},
} 

\newglossaryentry{FAA}
{
  type=\acronymtype,
  name={FAA},
  description={Federal Aviation Administration.
The U.S. Government agency that oversees and regulates the
airline industry in the U.S},
  first={Federal Aviation Administration (FAA)},
}

\newglossaryentry{GO}
{
  type=\acronymtype,
  name={GO},
  description={GBT Observing},
  first={GBT Observing (GO)},
}

\newglossaryentry{TLE}
{
  type=\acronymtype,
  name={TLE},
  description={Two-Line Element},
  first={Two-Line Element (TLE)},
}


\newglossaryentry{NAD83}
{
  type=\acronymtype,
  name={NAD83},
  description={North American Datum of 1983.
An earth-centered model for the Earth's surface based on 
  the Geodetic Reference System of 1980. The size and shape of the earth was 
  determined through measurements made by satellites and other sophisticated 
  electronic equipment; the measurements accurately represent the earth to 
  within two meters},
first={North American Datum of 1983 (NAD83)},
} 

\newglossaryentry{NAVD88}
{
  type=\acronymtype,
  name={NAVD88},
  description={The North American Vertical Datum of 1988},
first={North American Vertical Datum of 1988 (NAVD88)},
} 


\newglossaryentry{PRESTO}
{
  type=\acronymtype,
  name={PRESTO},
  description={PulsaR Exploration and Search TOolkit: A software package used
               to analyze pulsar observations},
  first={PulsaR Exploration and Search TOolkit (PRESTO)},
} 


\newglossaryentry{RDBE}
{
  type=\acronymtype,
  name={RDBE},
  description={A Roach Digital Backend, where ROACH is the core board
containing a large FPGA},
  first={Roach Digital Backend (RDBE)},
} 

\newglossaryentry{UTC}
{
  type=\acronymtype,
  name={UTC},
  description={Coordinated Universal Time.
               The mean solar time at 0\degree longitude},
  first={Coordinated Universal Time (UTC)},
} 


\newglossaryentry{MUSTANG}
{
  type=\acronymtype,
  name={MUSTANG},
  description={The MUltiplexed SQUID TES Array at Ninety~GHz
               80-100GHz bolometer receiver},
  first={The MUltiplexed SQUID TES Array at Ninety~GHz (MUSTANG)}
} 

\newglossaryentry{VNC}
{
  type=\acronymtype,
  name={VNC},
  description={Virtual Network Computer.
A GUI based system that is platform independent that allows
you to view the screen of one computer on a second computer.  This is very
useful for remote observing},
  first={Virtual Network Computer (VNC)},
}


%Glossary terms

\newglossaryentry{baseline}{name={baseline},
description={Baseline is a generic term usually taken to mean the instrumental
plus continuum bandpass shape in an observed spectrum, or changes in the
background level in a continuum observation}}


\newglossaryentry{Daisy}{name={daisy}, description={Observing pattern
known as the "daisy scan" based of the flower-like pattern it creates}}

\newglossaryentry{PFone}{name={PF1},
description={The first of two prime focus receivers for the GBT.  This 
receiver has four different bands: 290--395, 385--520, 510--690 and
680--920\,MHz}}

\newglossaryentry{PFtwo}{name={PF2},
description={The second of two prime focus receivers for the GBT.  This 
receiver covers 901--1230\,MHz}}

\newglossaryentry{Qband}{name={Q--band},
description={A region of the electromagnetic 
spectrum from 40--50\,GHz.}}

\newglossaryentry{Kuband}{name={Ku--band},
description={A region of the electromagnetic spectrum 
from 12--18\,GHz}}

\newglossaryentry{Cband}{name={C--band},
description={A region of the electromagnetic spectrum covering 4--8\,GHz
}}

\newglossaryentry{Sband}{name={S--band},
description={A region of the electromagnetic spectrum covering 2--4\,GHz}}

\newglossaryentry{Kaband}{name={Ka--band},
description={A region of the electromagnetic spectrum covering 26--40\,GHz}}

\newglossaryentry{Pband}{name={P--band},
description={A region of the electromagnetic spectrum covering 300--1000\,MHz.
Also known as the Ultra High Frequency (UHF) band in the U.S.  (Sometimes 
P--band is considered to be a narrow region around 408\,MHz, while A--band is
the region around 600\,MHz)}}

\newglossaryentry{Lband}{name={L--band},
description={A region of the electromagnetic spectrum covering 1--2\,GHz}}

\newglossaryentry{Xband}{name={X--band},
description={A region of the electromagnetic spectrum covering 8--12\,GHz}}

\newglossaryentry{Kband}{name={K--band},
description={A region of the electromagnetic spectrum covering 18--26\,GHz}}

\newglossaryentry{Wband}{name={W--band},
description={A region of the electromagnetic spectrum covering 75--111\,GHz}}


\newglossaryentry{A}{name={\ensuremath{\rm A}},
description={The number of air masses along the line of sight.  One 
air mass is defined as the total atmospheric column when looking at the
zenith},
sort=a}

\newglossaryentry{AFRack}{name={Analog Filter Rack},
description={A rack in the GBT IF system that provides contains filters to 
provide GUPPI and the DCR  with signals of the proper bandwidth}}


\newglossaryentry{Tsys}{name={\ensuremath{\rm T_{sys}}},
description={The total equivalent blackbody temperature brightness that the GBT sees.
Depending on usage, it may or may not include $T_{src}$},
sort=Temparature System}

\newglossaryentry{Trec}{name={\ensuremath{\rm T_{rec}}},
description={The equivalent blackbody temperature brightness that the GBT receiver 
contributes to the detected signal},
sort=Temparature Receiver}

\newglossaryentry{Tsrc}{name={\ensuremath{\rm T_{src}}},
description={The equivalent blackbody temperature brightness from the astronomical 
source},
sort=Temparature Source}

\newglossaryentry{voptical}{name={\ensuremath{\rm v_{optical}}},
description={The velocity of a source using the optical approximation of the
velocity--frequency relationship},
sort=velocity optical}

\newglossaryentry{vradio}{name={\ensuremath{\rm v_{radio}}},
description={The velocity of a source using the radio approximation of the
velocity--frequency relationship},
sort=velocity radio}

\newglossaryentry{vrelativistic}{name={\ensuremath{\rm v_{relativistic}}},
description={The velocity of a source using the relativistic definition of the
velocity--frequency relationship},
sort=velocity relativistic}

\newglossaryentry{vred}{name={\ensuremath{\rm v_{red}}},
description={The velocity of a source determined from the redshift (${\rm z}$)},
sort=velocity rec}

\newglossaryentry{sigmaw}{name={\ensuremath{\rm \sigma_w}},
description={The two--dimensional standard deviation of the GBT pointing error
resulting from the wind},
sort=sigma w}
 
\newglossaryentry{cw}{name={\ensuremath{\rm c_w}},
description={The speed of the wind},
sort=c w}
 
\newglossaryentry{tau}{name={\ensuremath{\rm \tau}},
description={The opacity of the atmosphere},
sort=tau}



\newglossaryentry{PROCNAME}{name={PROCNAME},
description={A GO FITS file keyword that contains the name of the Scan Type 
used in Astrid to obtain the data}}
 
\newglossaryentry{PROCSIZE}{name={PROCSIZE},
description={A GO FITS file keyword that contains the number of scans that 
are to be run as part of the Scan Type given by PROCNAME}}

\newglossaryentry{PROCSEQN}{name={PROCSEQN},
description={A GO FITS file keyword that contains the current number of 
scans done of the total scans given by PROCSIZE in a given Scan Type}}

\newglossaryentry{IFRack}{name={IF Rack},
description={A rack in the GBT IF system where the IF signal is distributed 
onto optical fibers and sent from the GBT receiver room to the GBT equipment
room where the backends are located.  A signal may also be sent directly to
the DCR}}


\newglossaryentry{CRRack}{name={Converter Rack},
description={A rack in the GBT IF system that recieves the signal from the
optical fibers (sent from the IF Rack), mixes the IF signal with LO2 and
LO3 references, and then distributes the IF signal to the various backends}}


\newglossaryentry{beamwidth}{name={beam--width},
description={The FWHM of the Gaussian response to the sky, the beam,
of the GBT}}

\newglossaryentry{DC}{name={Dynamic Corrections},
description={A system that uses temperature sensors located on the backup 
structure of the GBT to correct for deformations in the surface, and 
deformations that change the pointing and foucs of the GBT}}

\newglossaryentry{IDL}{name={IDL},
description={The Interactive Data Language program of
ITT Visual Information Solutions}}

\newglossaryentry{Pipeline}{name={Pipeline},
description={A data reduction scheme that allows the data to be reduced in
a pre-defined way if the data was taken in a specific manner}}

\newglossaryentry{noiseDiode}{name={noise diode},
description={A device with a known effective temperature that
is coupled to the telescope system to give a measure of system temperature
($T_{sys}$).  When the telescope is pointed on blank sky, the noise diode is
turned on and then off to determine the off-source system temperature.  This
device is also refered to as the \dq{Cal}}}

\newglossaryentry{fsw}{name={frequency switching},
description={A calibration method that obtains blank sky information while
keeping the telescope pointed at the object of interest.  The central frequency
is shifted such that the desired spectral lines appear at different locations
within the bandpass shape}}

\newglossaryentry{psw}{name={position switching},
description={A calibration method that involves observing an object of interest
period of time, and then moving the telescope to a blank sky region to obtain
the blank sky observations necessary for baseline subtraction.  Nodding is a
form of position switching.  Position switching is done via an observing
routine and is not setup in hardware unlike other switching schemes}}

\newglossaryentry{nodding}{name={nodding},
description={A dual-beam calibration method that involves observing an object
of interest with the first beam while the second beam is pointed at blank sky.
The roles of the beams are then switched by moving the telescope such that the
second beam is pointed towards the object, while the first beam observes blank
sky.  This allows position switching with no lost time while performing
off-source observations}}

\newglossaryentry{bsw}{name={beam switching},
description={The Ka-band (26-40~GHz) receiver is the only receiver that can
perform beam switching.  The switching can route the inputs of each feed to one
of two \dq{first amplifiers} which allows the short time-scale gain uctuations to be
removed from the data.  This type of switching is only recommended for
continuum observations.  Total power mode is recommended for Ka-band dual-beam
nodded observations using VEGAS as the backend}}

\newglossaryentry{polsw}{name={polarization switching},
description={This is only available for the L and X-band receivers.  During
an observation and at a rate of about once per second, the polarization of the
observation is switched between two orthogonal linear polarizations or the two
circular polarizations.  This switching method is used almost exclusively for
Zeeman measurements}}

\newglossaryentry{tpower}{name={total power},
description={Spectral-line observing typically requires differencing \dq{signal}
and \dq{reference} observations so as to remove the instrumental bandpass shape.
In total power observing, the reference observations are either separate scans
(as acquired with, for example, Astrid's OnOff or OffOn observing directives),
as separate integrations in an on-the-fly observations (for example, as edge
pixels in a map, or as separate integrations in some types of subreflector
nodding observations.  \dq{Switched Power}, the alternative to \dq{Total Power},
provides faster switching between signal and reference observations but, in
some cases, worse baseline shapes}}

\newglossaryentry{spower}{name={switched power},
description={Spectral-line observing typically requires differencing \dq{signal}
and \dq{reference} observations so as to remove the instrumental bandpass shape.
In switched power observing, the telescope typically will be frequency switching
(or, for some receivers, beam or polarization switching).  In these modes, the
reference and signal observations are acquired and stored separately in different
\dq{phases} within a switch period and, thus, within every integration.
\dq{Total Power}, the alternative to \dq{Switched Power}, provides slower switching
between signal and reference observations but, in some situations, better
baseline shapes}}

