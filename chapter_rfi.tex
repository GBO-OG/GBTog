\chapter{Radio Frequency Interference}\label{chap:rfi}

\glsfirst{RFI} can be a significant problem for some observations. The most
up to date information on the \gls{RFI} environment at the \gls{GBT} can be
found at:
\begin{itemize}
\item \htmladdnormallink
{http://www.gb.nrao.edu/IPG/}
{http://www.gb.nrao.edu/IPG/}
\end{itemize}

\noindent Useful resources, referenced from the above web page include
a list of known sources of \gls{RFI}: 
\begin{itemize}
\item \htmladdnormallink
{https://safe.nrao.edu/wiki/bin/view/GB/Projects/RFIReportsTable}
{https://safe.nrao.edu/wiki/bin/view/GB/Projects/RFIReportsTable}
\end{itemize}

\noindent and plots of RFI monitoring data:
\begin{itemize}
\item \htmladdnormallink
{http://www.gb.nrao.edu/IPG/rfiarchivepage.html}
{http://www.gb.nrao.edu/IPG/rfiarchivepage.html}
\end{itemize}

Every observer should check for known \gls{RFI} around their observing frequencies.
If you suspect that this could have a significant impact on your observations
you should contact your scientific support person to decide on an appropriate
course of action.

\topic{Mitigation of known RFI signals}

In some cases, it is possible to turn off a known \gls{RFI} source.
For example, there is an amateur transponder at about 432~MHz, which we can
request be shut down.  If there are known \gls{RFI} signals, the user
should discuss them with the scientific support person.  Given enough advance
warning (days to weeks), we may be able to have them shut down during the
observing.


