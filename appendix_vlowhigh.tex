\chapter{Usage of vlow and vhigh}\label{appendix:vlowhigh}

The configuration keywords {\bf vlow} and {\bf vhigh} give the range of
velocities of all sources to be observed. This information is used to set
various filters in the system that will simultaneously cover the required
range of velocity. Setting the velocity for each specific source is done
later in the \gls{SB}. For galactic sources where the range of velocities
is rather small it is usually best to set both vlow and vhigh to zero. 

When strong \gls{RFI} is present is it best not to use vlow and vhigh.  The use
of vlow and vhigh can cause the \gls{GBT} \gls{IF} system to have a larger
\gls{IF} bandwidth than is necessary for a single source.  This can let parts
of the \gls{IF} system be unnecessarily affected by \gls{RFI}. The observers
might need to reconfigure after each source if the change in velocity is larger
than the bandwidth of a filter.

An example of how vlow and vhigh can be used is as follows.  Suppose that
you are looking for water masers in extragalactic AGN.  Furthermore, lets
say that you are looking at 100 candidates with velocities from
$1000 ~{\rm km~s^{-1}}$ to $40000 ~{\rm km~s^{-1}}$.  Then you would
set vlow=1000.0 and vhigh=40000.0 and will not change the \gls{IF}
configuration when you change sources.
 
Note that if {\bf vdef}=\dq{Red} (i.e., redshift), then you must give the redshift 
parameter \dq{z} as the values for \dq{vlow} and \dq{vhigh} instead of velocity. 

Your scientific contact person can help you decide if you should use
{\bf vlow} and {\bf vhigh}.
