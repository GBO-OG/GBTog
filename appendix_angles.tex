%++++++++++++++++++++++++++++++++++++++++++++++++++++++++++++++++++++++++++++
\mychapter{A Note on Angle formats and units in Astrid and Catalogs}\label{appendix:angles}


There are two formats for angles in Observing Scripts and Catalogs:

\begin{itemize}
\item sexegesimal: e.g., hh:mm:ss[.ss], dd:mm:ss[.ss]
\item decimal numbers, e.g, ddd.ddd
\end{itemize}

When the quantity is RA or HA, an angle given in sexegesimal is hours, minutes, seconds of time.

For all other angle quantities (e.g., dec, az, el, glon, glat) an angle given in sexegesimal is degrees, minutes, seconds of arc.

In \dq{Location} and \dq{Offset} objects, a quantity given as a decimal number is always understood as being in units of degrees of arc, regardless of the type of unit.
However in Catalogs, a decimal number for RA or HA is assumed to be hours; for other angles (DEC, Az, El, Glon, Glat) it is degrees of arc.  

For example, if one is specifying an Offset object as in the following Astrid directive:
\begin{verbatim}
    OnOff( "3C286", Offset("J2000", "00:04:00", 0.5), 60, "1")
\end{verbatim}
The offset will be 4 minutes of time in RA and 0.5 degrees of arc in DEC.
(The coordinate mode \dq{J2000} means the coordinate pair is (RA,DEC), hence sexegesimal RA is assumed to be in hours.)

Alternately if one says:
\begin{verbatim}
    OnOff( "3C286", Offset("J2000", 1.0, 0.5), 60, "1")
\end{verbatim}
The offset will be one degree of arc in RA and 0.5 degrees of arc in DEC.

\newpage


%++++++++++++++++++++++++++++++++++++++++++++++++++++++++++++++++++++++++++++